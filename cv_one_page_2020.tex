\documentclass[11pt, a4paper]{moderncv}
\moderncvtheme[purple]{classic}
\usepackage[utf8]{inputenc}
\usepackage[top=1.1cm, bottom=1.1cm, left=2cm, right=2cm]{geometry}
% Largeur de la colonne pour les dates
\setlength{\hintscolumnwidth}{2.5cm}
\firstname{Charlotte}
\familyname{Thomas}
\title{Étudiante / DevOps Java Backend}
\address{8 Rue Victor Basch}{00000 Une ville}
\email{charlotte@salondesdevs.fr}
\mobile{06 00 00 00 00}
\extrainfo{18 ans}
\quote{`I think I can safely say that nobody understands Quantum Mechanics\newline{}Richard Feynman`}
\begin{document}
\maketitle
\vspace*{-2.5\baselineskip}
\section{Scolaire}
\cventry{2020-2021}{Classe Préparatoire aux Grandes Écoles | MPSI}{Lycée Montaigne}{Bordeaux (33)}{}{}
\cventry{2017 - 2020}{Baccalauréat Scientifique}{Lycée Gustave Eiffel}{Bordeaux (33)}{Mention Bien}{}
\section{Expérience en Développement}
\cventry{2020 --}{Omega}{Développement}{}{}{Aide au Développement d'Omega, un OS Open Source de calculatrice \newline}
\cventry{2019 --}{STARFLEET International}{Développement et Administration Système}{USS Versailles}{Section des Sciences}{}
\cventry{2019 --}{Star Trek French Club}{Développement et Administration Système}{}{}{Développement et exploitation d'un système de génération de quizz automatique en Java \newline}
\cventry{Depuis 2013}{Personnel}{Développement et Administration Système}{}{}{Développement d'une variété de projets libres}
\section{Expérience Professionelle}
\cventry{Décembre 2017}{Stage d'Observation}{ESI de Bordeaux}{Direction Générale des Finances Publiques}{Bordeaux (33)}{
\begin{itemize}
 \item Visite du pôle sécurité 
 \item Visite du pôle réseau
 \item Visite du pôle intégration
\end{itemize}}
\section{Compétences en informatique}
\cvitem{\underline{OS}}{CentOS, Debian}
\cvitem{\underline{Développement}}{Java, Python, HTML, CSS, Javascript, PHP, LaTeX, Git}
\cvitem{\underline{Admin. services}}{Apache2, MySQL}
\cvlanguage{\underline{Anglais}}{B2}{}
\section{Centres d'intérêt}
\cvitem{}{Développement, Administration Système, Mathématiques et Sciences Physiques, Star Trek}{}{}
\end{document}
