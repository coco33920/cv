\documentclass[11pt, a4paper, hidelinks]{moderncv}
\moderncvtheme[purple]{classic}
\usepackage[utf8]{inputenc}
\usepackage{moderntimeline}
\usepackage[top=1.1cm, bottom=1.1cm, left=2cm, right=2cm]{geometry}
% Largeur de la colonne pour les dates
\setlength{\hintscolumnwidth}{2.5cm}
\firstname{Charlotte}
\familyname{Thomas}
\title{Étudiante | Développeuse}
\address{}{Rennes}
\email{contact@nwa2coco.fr}
\mobile{06 00 00 00 00}
\social[github]{coco33920}
\social[twitter]{coco33920}
\extrainfo{Elle/She/They 20F}
\begin{document}
\maketitle
\tlmaxdates{2012}{2025}
\vspace*{-2.5\baselineskip} 
\section{Scolaire}
\tlcventry{2022}{2023}{L2 Informatique}{Université de Rennes}{Rennes (35)}{}{}
\tlcventry{2021}{2022}{Classe Préparatoire aux Grandes Écoles | MP}{Lycée Montaigne}{Bordeaux (33)}{}{}
\tlcventry{2020}{2021}{Classe Préparatoire aux Grandes Écoles | MPSI}{Lycée Montaigne}{Bordeaux (33)}{}{}
\tlcventry{2017}{2020}{Baccalauréat Scientifique}{Lycée Gustave Eiffel}{Bordeaux (33)}{Mention Bien}{}
\section{Expérience en Développement}
\tlcventry{2021}{0}{BaguetteSharp}{Développement}{}{}{
    \begin{itemize}
        \item Développement d'un Interpreter et REPL multiplateforme pour un langage éxotique
        \item Sans bibliothèque pour la partie language
        \item Originellement pour le TIPE ENS de la session des concours 2022
        \item Interface Web pour une utilisation simplifiée
        \item Licence MIT sur GitHub (coco33920/ocaml-baguettesharp-interpreter)
    \end{itemize}
}
\tlcventry{2019}{0}{Automatic Report System}{Écosystème}{}{}{
    \begin{itemize}
        \item Développement d'un Écosystème de rapports automatique
        \item Développé en Java, utilisé brièvement pour STARFLEET International
        \item Architecture client-server sécurisé
        \item Licence MIT sur GitHub (pin sur mon profil)
    \end{itemize}
}
\tlcventry{2013}{0}{Personnel}{Développement et Administration Système}{}{}{Développement d'une variété de projets libres}
\tlcventry{2020}{2021}{Omega}{Développement}{}{}{Aide au Développement d'Omega, un OS Open Source de calculatrice}
\section{Expérience Professionelle}
\tlcventry{2021}{2021}{Vacataire}{Cellule Informatique Départementale (CID33), \textit{ESI de Bordeaux}}{Direction Générale des Finances Publiques}{Bordeaux (33)}{
    \begin{itemize}
        \item Inventaire du parc informatique de la DRFIP Nouvelle Aquitaine
        \item Mise en stock et sécurisation du matériel informatique
        \item Dépannage divers des agents
    \end{itemize}
}
\tlcventry{2017}{2017}{Stage d'Observation}{ESI de Bordeaux}{Direction Générale des Finances Publiques}{Bordeaux (33)}{
\begin{itemize}
 \item Visite du pôle sécurité 
 \item Visite du pôle réseau
 \item Visite du pôle intégration
\end{itemize}}
\section{Compétences en informatique avancés}
\cvitem{\underline{OS}}{Linux RPM, Linux DEB}
\cvitem{\underline{Langages}}{Java, OCaml, SQL, JavaScript, \LaTeX}
\cvitem{\underline{Framework}}{Java Spark, JS Vue, JS Aurelia}
\cvitem{\underline{Admin.}}{Apache2, MySQL, NodeJS}
\cvlanguage{\underline{Anglais}}{B2}{}
\section{Centres d'intérêt}
\cvitem{}{Développement, Écriture de webnovel, Mathématiques et Sciences Physiques, Star Trek}{}{}
\end{document}
