\documentclass[11pt, a4paper, colorlinks]{moderncv}
\moderncvtheme[purple]{classic}
\usepackage[utf8]{inputenc}
\usepackage{moderntimeline}
\usepackage[top=1.1cm, bottom=1.1cm, left=2cm, right=2cm]{geometry}
% Largeur de la colonne pour les dates
\setlength{\hintscolumnwidth}{2.5cm}
\firstname{Charlotte}
\familyname{Thomas}
\title{Étudiante | Développeuse}
\address{}{35000 Rennes}
\email{contact@nwa2coco.fr}
\mobile{06 00 00 00 00}
\social[github]{coco33920}
\social[twitter]{coco33920}
\social[HAL][https://cv.hal.science/coco33920]{HAL coco33920}
\extrainfo{Elle/She/They 20F}
\begin{document}
\maketitle
\tlmaxdates{2012}{2025}
\vspace*{-2.5\baselineskip} 
\section{Scolaire}
\tlcventry{2023}{2023}{Stage de Recherche}{IRISA/INRIA Rennes}{Équipe PACAP}{OFast3D}{
    \begin{itemize}
        \item Transformation mathématique de GCode.
    \end{itemize}
}
\tlcventry{2022}{2023}{L2 Informatique}{Université de Rennes}{Rennes (35)}{}{}
\tlcventry{2020}{2022}{Classe Préparatoire aux Grandes Écoles | MPSI/MP}{Lycée Montaigne}{Bordeaux (33)}{}{}
\section{Projets Principaux}
\tlcventry{2022}{0}{\href{https://boulangerie.baguettesharp.fr}{Boulangerie}}{BaguetteSharp Package Manager}{}{}{
    \begin{itemize}
        \item \href{https://github.com/coco33920/boulangerie}{Page GitHub}
    \end{itemize}
}
\tlcventry{2021}{0}{\href{https://www.baguettesharp.fr}{BaguetteSharp}}{Théorie des Langages/Développement}{}{}{
    \begin{itemize}
        \item Développement d'un Interpreter et REPL multiplateforme pour un langage ésotérique
        \item Publié sous GPLv3 sur \href{https://github.com/coco33920/ocaml-baguettesharp-interpreter}{GitHub}
    \end{itemize}
}
\tlcventry{2022}{0}{STARFinder}{Développement}{}{}{
    \begin{itemize}
        \item Création d'un REPL pour trouver \textit{votre} arrêt de bus grâce au pouvoir de la logique!
        \item GPLv3 sur \href{https://github.com/coco33920/STARFinder}{GitHub}, fait pour apprendre le Scala et son intégration avec Java
    \end{itemize}
}
\section{Autres projets}
\tlcventry{2020}{2022}{Omega}{OS pour Numworks Calculatrices}{Développeuse}{}{Page \href{https://github.com/Omega-Numworks/Omega}{GitHub}}
\tlcventry{2019}{2021}{Automatic Report System}{Écosystème}{}{}{
    \begin{itemize}
        \item Développement d'un Écosystème de rapports automatique développé pour STARFLEET International
        \item Licence MIT sur GitHub
    \end{itemize}
}
\tlcventry{2013}{0}{Personnel}{Développement et Administration Système}{}{}{Développement d'une variété de projets libres, \href{https://github.com/coco33920}{Profil GitHub}}
\section{Expériences Professionelles}
\tlcventry{2021}{2021}{Vacataire}{Cellule Informatique Départementale (CID33), \textit{ESI de Bordeaux}}{Direction Générale des Finances Publiques}{Bordeaux (33)}{}
\section{Concours et Récompenses}
\cvitem{\textit{Prologin}}{Finales Régionales 2019,2020,2021}
\cvitem{\textit{Olympiades}}{10ème aux Olympiades Académiques de Mathématiques de Bordeaux (1ère)}
\section{Compétences}
\cvitem{\textit{OS}}{Linux RPM, Linux DEB}
\cvitem{\textit{Langages}}{Java, OCaml, SQL, JavaScript, \LaTeX}
\cvitem{\textit{Admin.}}{Apache2, MySQL, NodeJS}
\cvlanguage{\textit{Anglais}}{Couramment}{}
\section{Centres d'intérêt}
\cvitem{}{Théorie des Langages Formels, Informatique, Écriture de webnovel, Mathématiques, Science Fiction}{}{}
\end{document}
